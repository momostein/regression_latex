\section{Inleiding}

\section{Teamwerk}
\subsection{Git}
Git is een \textit{Version Control System} (VCS) oftewel een versiebeheersysteem. De oorspronkelijke ontwikkelaar van git is Linus Torvalds in 2005.\cite{init_git} Ondertussen werken er al veel meer ontwikkelaars aan.

In een zogenoemde \textit{Git repository} worden alle verandering van code onthouden. Deze veranderingen slaagt men als programmeur op per nieuwe revisie. Deze revisies worden allemaal bijgehouden op chronologische volgorde. Zo kan men teruggaan naar een vorige revisie en deze vergelijken met de nieuwere revisie waar een fout in zou zitten.
\subsection{Github}
Github is een online platform voor versiebeheer via Git. Via Github kunnen programmeurs hun \textit{repository} online opslaan. Zo kunnen zij altijd en overal aan hun code en deze ook snel delen met andere programmeurs.

\section{JSON}
\lstinputlisting[language=JavaScript]{code/rules_d2.json}


\section{Boomstructuur}
\section{Regex}
De beslissing of de prijs wordt opgeslagen in een string. Om deze data snel en foutloos uit deze string te halen gebruiken we regular expressions oftewel regex. Een zogenoemde regular expression is een reeks karakters die samen een patroon definiëren. Zo'n patroon wordt gebruikt om bepaalde dingen te zoeken, om speciefieke dingen te veranderen of om een text input te controleren. \cite{wiki:regex}

\subsection{Beslissing}
\begin{lstlisting}[language=JavaScript]
{
    "name": "Model_B3 > 0.5",
    "children": [{...}, {...}]
}    
\end{lstlisting}

\subsection{Prijs}
\begin{lstlisting}[language=JavaScript]
{
    "name": "267 of Price",
    "children": null
}    
\end{lstlisting}
\section{Inladen van de structuur}

\section{Testen}
\begin{tikzpicture}
\begin{axis}[
xlabel={$Boom diepte $},
ylabel={$Gem. ladingstijd (s)$}]
\addplot table [x=Tree depth, y=Evarage load time, col sep=semicolon] {data.csv};
\end{axis}
\end{tikzpicture}
\begin{tikzpicture}
\begin{axis}[
xlabel={$Boom diepte $},
ylabel={$Gem. uitvoertijd (s)$}]
\addplot table [x=Tree depth, y=Everage estimate time, col sep=semicolon] {data.csv};
\end{axis}
\end{tikzpicture}
