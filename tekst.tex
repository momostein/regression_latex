\section{Inleiding}

\section{Teamwerk}
Één programma schrijven met meerdere programmeurs is niet simpel. Daarom gebruiken wij twee belangrijke tools. Met deze tools konden wij ons meer focussen op het programma zelf en minder op het oplossen van randproblemen.

\subsection{Git}
De eerste tool die wij gebruiken is Git.
Git is een \textit{Version Control System} (VCS) oftewel een versiebeheersysteem. De oorspronkelijke ontwikkelaar van git is Linus Torvalds in 2005.\cite{git:init} Ondertussen werken er al veel meer ontwikkelaars aan.

In een zogenoemde \textit{Git repository} worden alle verandering van code onthouden. Deze veranderingen slaagt men als programmeur op per nieuwe revisie. Deze revisies worden allemaal bijgehouden op chronologische volgorde. Zo kan men teruggaan naar een vorige revisie en deze vergelijken met de nieuwere revisie. Zo kan men de oorzaak van een fout sneller opsporen.

\subsection{Github}
De website \textit{www.github.com}, oftewel Github, is de tweede tool.
Deze website is een online platform voor versiebeheer via Git. Via Github kunnen programmeurs hun \textit{repository} online opslaan.\cite{git:hello_world} Zo kunnen zij altijd en overal aan hun code en deze ook snel delen met andere programmeurs.

\clearpage
\section{JSON}
\lstinputlisting[language=JavaScript]{code/rules_d2.json}


\section{Boomstructuur}
\subsection{Tree-object}
\subsection{TreeNode-object}

\section{Inladen van de structuur}

\section{Regex}
De beslissing of de prijs wordt opgeslagen in een string. Om deze data snel en foutloos uit deze string te halen gebruiken we regular expressions oftewel regex. Een zogenoemde regular expression is een reeks karakters die samen een patroon definiëren. Zo'n patroon wordt gebruikt om bepaalde dingen te zoeken, om speciefieke dingen te veranderen of om inputs te controleren.\cite{wiki:regex}

\subsection{Prijs}
\begin{lstlisting}[language=JavaScript]
{
    "name": "267 of Price",
    "children": null
}    
\end{lstlisting}
\subsection{Beslissing}

\begin{lstlisting}[language=JavaScript]
{
    "name": "Model_B3 > 0.5",
    "children": [{...}, {...}]
}    
\end{lstlisting}

\section{Testen}
\begin{table}
    \centering
    \csvautotabular{data.csv}
    \caption{Testresultaten}
    \label{tab:test_results}
\end{table}

\begin{figure}[ht]
    \centering
    \begin{tikzpicture}
        \begin{axis}[
            width=0.7\textwidth,
            xlabel={$Boom diepte $},
            ylabel={$Gem. ladingstijd (s)$}]
            \addplot table [x=Depth, y=Load Time, col sep=comma] {data.csv};
        \end{axis}
    \end{tikzpicture}

    \begin{tikzpicture}
        \begin{axis}[
            width=0.7\textwidth,
            xlabel={$Boom diepte $},
            ylabel={$Gem. uitvoertijd (s)$}]
            \addplot table [x=Depth, y=Estimate Time, col sep=comma] {data.csv};
        \end{axis}
    \end{tikzpicture}
    
    \caption{Grafieken Testresultaten}
    \label{fig:test_result}
\end{figure}

% We fixen de positie van deze figuur nog wel
