%%%%%%%%%%%%%%%%%%%%%%%%%%%%%%%%%%%%%%%%%
% University Assignment Title Page
% LaTeX Template
% Version 1.0 (27/12/12)
%
% This template has been downloaded from:
% http://www.LaTeXTemplates.com
%
% Original author:
% WikiBooks (http://en.wikibooks.org/wiki/LaTeX/Title_Creation)
%
% License:
% CC BY-NC-SA 3.0 (http://creativecommons.org/licenses/by-nc-sa/3.0/)
%
%
% ============Instellingen:============
%
%\documentclass[12pt]{article}
\documentclass[12pt,a4paper]{article}
\special{papersize=210mm,297mm}
\usepackage[top=2.5cm, bottom=3cm, left=2.5cm, right=2.5cm]{geometry}
%\usepackage[english]{babel}
\usepackage[dutch]{babel}
\selectlanguage{dutch}
\usepackage[utf8x]{inputenc}
%\usepackage{amsmath}
\usepackage{amsmath, amsthm, amssymb, amsfonts}
\usepackage{graphicx}
\usepackage{float}
\usepackage{caption}
\usepackage{subcaption}
\usepackage{wrapfig}
\usepackage{framed}
\usepackage{stfloats}
\usepackage{pdflscape}
% Maakt dat je landscape kan werken.
% Zie: https://www.ctan.org/pkg/pdflscape
\usepackage{csquotes}
\usepackage{multicol}
\usepackage{pgfplots}
\usepackage{csvsimple}
\usepackage{filecontents}
\setlength{\marginparwidth}{2cm}

\usepackage[colorinlistoftodos]{todonotes}
\usepackage{color}
% \colorbox{declared-color}{TO DO / ...}
% \textcolor{declared-color}{text}
%\usepackage{fontspec}
%\setmainfont{Calibri.ttf}

%\parskip = \baselineskip
% Bevel verplaatst na table of content.
% Vorig bevel: voeg een extra blanco lijn toe na elke paragraaf

\setlength{\parindent}{0em}
\usepackage{enumitem}

\captionsetup[figure]{name=Figuur}
% Om Nederlandse figuurnamen te hebben
\captionsetup[table]{name=Tabel}
%\captionsetup{font=footnotesize} toegevoegd om de bijschriften ook in een kleiner lettertype te hebben.
\captionsetup{font=footnotesize}

\addto\captionsenglish{% Replace "english" with the language you use
  \renewcommand{\contentsname}%
    {Inhoudstafel}%
}

%
% ============ Settings voor source code blocks ============
%
%

\usepackage{listings} % Om code te laten zien
\usepackage{xcolor}

\definecolor{commentsColor}{rgb}{0.497495, 0.497587, 0.497464}
\definecolor{keywordsColor}{rgb}{0.000000, 0.000000, 0.635294}
\definecolor{stringColor}{rgb}{0.558215, 0.000000, 0.135316}

\lstset{ %
  backgroundcolor=\color{white},   % choose the background color; you must add \usepackage{color} or \usepackag{xcolor}
  basicstyle=\footnotesize,        % the size of the fonts that are used for the code
  breakatwhitespace=false,         % sets if automatic breaks should only happen at whitespace
  breaklines=true,                 % sets automatic line breaking
  captionpos=b,                    % sets the caption-position to bottom
  commentstyle=\color{commentsColor}\textit,    % comment style
  deletekeywords={...},            % if you want to delete keywords from the given language
  escapeinside={\%*}{*)},          % if you want to add LaTeX within your code
  extendedchars=true,              % lets you use non-ASCII characters; for 8-bits encodings only, does not work with UTF-8
  frame=tb,	                   	   % adds a frame around the code
  keepspaces=true,                 % keeps spaces in text, useful for keeping indentation of code (possibly needs columns=flexible)
  keywordstyle=\color{keywordsColor}\bfseries,       % keyword style
  language=Python,                 % the language of the code (can be overrided per snippet)
  otherkeywords={*,...},           % if you want to add more keywords to the set
  numbers=left,                    % where to put the line-numbers; possible values are (none, left, right)
  numbersep=5pt,                   % how far the line-numbers are from the code
  numberstyle=\tiny\color{commentsColor}, % the style that is used for the line-numbers
  rulecolor=\color{black},         % if not set, the frame-color may be changed on line-breaks within not-black text (e.g. comments (green here))
  showspaces=false,                % show spaces everywhere adding particular underscores; it overrides 'showstringspaces'
  showstringspaces=false,          % underline spaces within strings only
  showtabs=false,                  % show tabs within strings adding particular underscores
  stepnumber=1,                    % the step between two line-numbers. If it's 1, each line will be numbered
  stringstyle=\color{stringColor}, % string literal style
  tabsize=2,	                   % sets default tabsize to 2 spaces
  title=\lstname,                  % show the filename of files included with \lstinputlisting; also try caption instead of title
  columns=fixed                    % Using fixed column width (for e.g. nice alignment)
}

\lstdefinelanguage{JavaScript}{
  keywords={typeof, new, true, false, catch, function, return, null, catch, switch, var, if, in, while, do, else, case, break},
  keywordstyle=\color{keywordsColor}\bfseries,
  ndkeywords={class, export, boolean, throw, implements, import, this},
  ndkeywordstyle=\color{darkgray}\bfseries,
  identifierstyle=\color{black},
  sensitive=false,
  comment=[l]{//},
  morecomment=[s]{/*}{*/},
  commentstyle=\color{commentsColor}\ttfamily,
  stringstyle=\color{stringColor}\ttfamily,
  morestring=[b]',
  morestring=[b]"
}

%
% ============ BEGIN VAN HET DOCUMENT ============
%
%
\begin{document}

%Titelpagina....
% Titelpagina

\begin{titlepage}

\newcommand{\HRule}{\rule{\linewidth}{0.5mm}} % Defines a new command for the horizontal lines, change thickness here

\center % Center everything on the page
 
%----------------------------------------------------------------------------------------
%	HEADING SECTIONS
%----------------------------------------------------------------------------------------

\textsc{\LARGE KU Leuven}\\[1.5cm] % Name of your university/college


%----------------------------------------------------------------------------------------
%	TITLE SECTION
%----------------------------------------------------------------------------------------

\HRule \\[0.4cm]
{ \huge \bfseries Project 1: Regressieboom}\\[0.4cm] % Title of your document
\HRule \\[1.5cm]
 
%----------------------------------------------------------------------------------------
%	AUTHOR SECTION
%----------------------------------------------------------------------------------------

\begin{minipage}{0.4\textwidth}
\begin{flushleft} \large
\emph{Student:}\\
Brecht \textsc{Ooms} \\ % Your name
Sander \textsc{Ouderits} % Your name



\end{flushleft}
\end{minipage}
~
\begin{minipage}{0.4\textwidth}
\begin{flushright} \large
\emph{Begeleiders:} \\
Peter \textsc{Karsmakers} % second supervisor's Name


\end{flushright}
\end{minipage}\\[2cm]

% If you don't want a supervisor, uncomment the two lines below and remove the section above
%\Large \emph{Author:}\\
%John \textsc{Smith}\\[3cm] % Your name

%----------------------------------------------------------------------------------------
%	DATE SECTION
%----------------------------------------------------------------------------------------

{\large \today}\\[2cm] % Date, change the \today to a set date if you want to be precise

%----------------------------------------------------------------------------------------
%	LOGO SECTION
%----------------------------------------------------------------------------------------

\includegraphics[width=2.5in]{logokuleuven.png}\\[1cm] % Include a department/university logo - this % will require the graphicx package
 
%----------------------------------------------------------------------------------------

\vfill % Fill the rest of the page with whitespace

\end{titlepage}


\small
\tableofcontents
\newpage
\listoffigures
\newpage
\listoftables
% Volgend bevel staat pas hier, want anders staat er een blanco lijn na elke lijn in de inhoudstafel en de lijsten van figuren en tabellen:
\parskip = \baselineskip
\newpage
\Large
\textbf{Lijst van afkortingen}
\newline
\small
\begin{table}[ht]
% https://nl.wikibooks.org/wiki/LaTeX/Tabellen
\small
\begin{tabular}{ll} 
VCS         & Version Control System \\
JSON        & JavaScript Object Notation \\
ifstream    & input file stream \\
Regex       & Regular Expressions \\
Abs.        & Absolute \\
Rel.        & Relative \\
VS          & Visual Studio \\
OC          & Overclocked \\
CPU         & Central Processing Unit \\
% AM &Amplitudemodulatie (Amplitude Modulation) \\
% DSN &Deep Space Network\\
% FM &Frequentiemodulatie (Frequency Modulation) \\
% NASA &National Aeronautics and Space Administration \\
\end{tabular}
\end{table}

\newpage

%\begin{multicols}{2}
\section{Inleiding}

\section{Teamwerk}
Één programma schrijven met meerdere programmeurs is niet simpel. Daarom gebruiken wij twee belangrijke tools. Met deze tools konden wij ons meer focussen op het programma zelf en minder op het oplossen van randproblemen.

\subsection{Git}
De eerste tool die wij gebruiken is Git.
Git is een \textit{Version Control System} (VCS) oftewel een versiebeheersysteem. De oorspronkelijke ontwikkelaar van git is Linus Torvalds in 2005.\cite{git:init} Ondertussen werken er al veel meer ontwikkelaars aan.

In een zogenoemde \textit{Git repository} worden alle verandering van code onthouden. Deze veranderingen slaagt men als programmeur op per nieuwe revisie. Deze revisies worden allemaal bijgehouden op chronologische volgorde. Zo kan men teruggaan naar een vorige revisie en deze vergelijken met de nieuwere revisie. Zo kan men de oorzaak van een fout sneller opsporen.

\subsection{Github}
De website \textit{www.github.com}, oftewel Github, is de tweede tool.
Deze website is een online platform voor versiebeheer via Git. Via Github kunnen programmeurs hun \textit{repository} online opslaan.\cite{git:hello_world} Zo kunnen zij altijd en overal aan hun code en deze ook snel delen met andere programmeurs.

\clearpage
\section{JSON}
\lstinputlisting[language=JavaScript]{code/rules_d2.json}


\section{Boomstructuur}
\subsection{Tree-object}
\subsection{TreeNode-object}

\section{Inladen van de structuur}

\section{Regex}
De beslissing of de prijs wordt opgeslagen in een string. Om deze data snel en foutloos uit deze string te halen gebruiken we regular expressions oftewel regex. Een zogenoemde regular expression is een reeks karakters die samen een patroon definiëren. Zo'n patroon wordt gebruikt om bepaalde dingen te zoeken, om speciefieke dingen te veranderen of om inputs te controleren.\cite{wiki:regex}

\subsection{Prijs}
\begin{lstlisting}[language=JavaScript]
{
    "name": "267 of Price",
    "children": null
}
\end{lstlisting}
\subsection{Beslissing}

\begin{lstlisting}[language=JavaScript]
{
    "name": "Model_B3 > 0.5",
    "children": [{...}, {...}]
}
\end{lstlisting}

\section{Testen}
In dit deel wordt getest hoe goed de pijsbepaling gebeurt bij bepaalde boomdieptes. Er wordt ook getest wat de invloed is van de boomdiepte op de tijd die het programma nodig heeft om de boom in te laden en om de prijs te bepalen. In onderstaande tabel staan de resultateen voor een boomdiepte van één tof vijf.
\begin{table}
    \centering
    \csvautotabular{data.csv}
    \caption{Testresultaten}
    \label{tab:test_results}
\end{table}

De inlaad tijd van de boom is duidelijk afhankelijk van de boomdiepte. Bij een diepere boom zien we een duidelijke stijging in inlaadtijd. Met de grote-O-notatie is dit een complexiteit van $ O({2^n}) $. In dit geval is het geen worst case scenario omdat de boom niet overal vertakt tot beneden.

Bij het bepalen van de prijs zien we ook dat het langer duurt bij een diepere boom.

Bij het testen gebruiken we alleen orgels waarvan we de prijs kennen. Het programma maakt dan de schatting van de prijs. Deze prijs wordt dan vergeleken met echte prijs. Bij het nemen van de absolutewaarde van de fout en hier het gemmiddelde van te nemen over alle orgels krijgen we de absolute fout die te vinden is in de tabel. Het is heel dijdelijk dat deze fout kleiner wordt als de boom dieper wordt. Dit komt natuurlijk door dat er meer beslissing worden genomen en dus een betere prijsschating kan worden gedaan. Als we dan naar de procentuele fout gaan kijken, is deze bij een diepte van 5 heel klein. Daaruit kunnen we besluiten dat we bij een diepte van 5 een precise bepaling hebben van de pijs.
\begin{figure}[ht]
    \centering
    \begin{tikzpicture}
        \begin{axis}[
            width=0.7\textwidth,
            xlabel={$Boom diepte $},
            ylabel={$Gem. ladingstijd (s)$}]
            \addplot table [col sep=comma, x=Tree depth , y=Load time (s)] {data.csv};
        \end{axis}
    \end{tikzpicture}

    \begin{tikzpicture}
        \begin{axis}[
            width=0.7\textwidth,
            xlabel={$Boom diepte $},
            ylabel={$Gem. uitvoertijd (s)$}]
            \addplot table [ col sep=comma, x=Tree depth, y=Estimate time (s)] {data.csv};
        \end{axis}
    \end{tikzpicture}

    \caption{Grafieken Testresultaten}
    \label{fig:test_result}
\end{figure}

% We fixen de positie van deze figuur nog wel

%\input{voorbeeld.tex}
%\end{multicols}

\scriptsize
\newpage
\begin{flushleft}
\bibliographystyle{IEEEtran}
\bibliography{bronnen} %Je mag deze file een andere naam geven, maar de extensie moet 'bib' zijn.
\end{flushleft}

\end{document}
